\documentclass[french]{article}
\usepackage[T1]{fontenc}
\usepackage[utf8]{inputenc}
\usepackage{lmodern}
\usepackage[a4paper]{geometry}
\usepackage{babel}
\usepackage{listings}             % Include the listings-package

\title{Recherche des structures secondaires d’une chaîne d’ADN\\
\large Projet 2I003}
%\title{Recherche des structures secondaires d’une chaîne d’ADN}
\author{LI Mengda, GE Zhichun}


\begin{document}
\lstset{language=Python}	

\maketitle

\section{Exercice 1}

\subsection{}
\label{subsec:q1}
Comme un nucléotide ne peut former une paire avec lui même, donc 

\begin{center}
$\forall i\in\{1,...,n\}$, \quad S$_{i,i}$ = \{\} et E$_{i,i}$ = 0
\par\end{center}

\subsection{}
\label{subsec:q2}

	\subsubsection{}
	Si ni i, ni j ne sont couplés dans S$_{i,j}$ , alors
	%{\centering S$_{i,j}$ = S$_{i+1,j-1}$\par}
	\begin{center}
	S$_{i,j}$ = S$_{i+1,j-1}$, \quad E$_{i,j}$ = E$_{i+1,j-1}$
	\par\end{center}

	\subsubsection{}
	Si j n'est pas couplée dans S$_{i,j}$ , alors
	\begin{center}
	S$_{i,j}$ = S$_{i,j-1}$,  \quad E$_{i,j}$ = E$_{i,j-1}$
	\par\end{center}

	\subsubsection{}
	Si $\left(i,j\right)\in S_{i,j}$, alors
	\begin{center}
	$S_{i,j} $  = S$_{i+1,j-1}\, \cup \, \left(i,j\right)$, \quad  E$_{i,j}$  = E$_{i+1,j-1}+ 1$
	\par\end{center}

	\subsubsection{}
	Si $\left(k,j\right)\in S_{i,j}$ avec $k\in\left\{ i+1,...,j-1\right\} $, alors
	\[
	S_{i,j}= S_{i,k-1}\cup S_{k,j} \quad E_{i,j} = E_{i,k-1} + E_{k,j} 
	\]

\subsection{}
	\subsubsection{}
		Par la question \ref{subsec:q2}, résumons en distinguant tous les cas possibles:
	
		\begin{enumerate}  
		\item \label{itm:1}
			 Soit ni i, ni j ne sont couplés dans S$_{i,j}$ , alors E$_{i,j}$ = E$_{i+1,j-1}$, \\
			sinon on a toujours E$_{i,j} \geq E_{i+1,j-1}$
		\item Soit j n'est pas couplée dans S$_{i,j}$ , alors  E$_{i,j}$ = E$_{i,j-1}$, \\
			sinon on a toujours E$_{i,j} \geq E_{i,j-1}$
	%	\item Soit i n'est pas couplée dans S$_{i,j}$ , alors  E$_{i,j}$ = E$_{i+1,j}$
		\item Soit j est couplé avec i, c'est à dire $\left(i,j\right)\in S_{i,j}$, alors E$_{i,j}$  = E$_{i+1,j-1}+1$
		\item Soit j est couplé avec un $k\in\left\{ i+1,...,j-1\right\} $, alors $E_{i,j} = E_{i,k-1} + E_{k,j}$,\\ sinon $E_{i,j} \geq E_{i,k-1} + E_{k,j}$ car $S_{i,j} \supset \, S_{i,k-1}\cup S_{k,j}\quad \forall k \in\left\{ i+1,...,j-1\right\}$
		\end{enumerate}
		
		Soit la fonction e :$ \{1, ... , n\} \times{} \{1, ... , n\} \rightarrow{} \{0, 1\}$ qui vaut 1 ssi i et j peuvent être couplés. \par
		\begin{description}
		\item
		\underline{Dans le cas  \ref{itm:1} et cas 3, $E_{i,j} = E_{i+1,j-1} + e(i,j)$ }  et on a toujours \underline{$E_{i,j} \geq E_{i+1,j-1} + e(i,j)$ }car 
			\begin{itemize} 
			\item si j est couplé avec i, $ e(i,j) = 1$, E$_{i,j}  = E_{i+1,j-1}+1 = E_{i+1,j-1}+ e(i,j) $, 
			\item sinon $ e(i,j) = 0$, on a toujours E$_{i,j} \geq E_{i+1,j-1} = E_{i+1,j-1}+ e(i,j)$ par \ref{itm:1}.
			\end{itemize}	
		\item
		\underline{Dans le cas 2, $E_{i,j} = E_{i,j-1}$} et on a toujours \underline{$E_{i,j} \geq E_{i,j-1}$} car S$_{i,j} \supset S_{i,j-1}$
		\item
		Par le cas 4, $\forall k \in\left\{ i+1,...,j-1\right\} \; E_{i,j} \geq E_{i,k-1} + E_{k,j}$. Donc \underline{ $E_{i,j} \geq  \displaystyle \max_{i<k<j}  E_{i,k-1} + E_{k,j}$ }.
		\\Et
		s'il existe $k_{0}\in\left\{ i+1,...,j-1\right\}$ tel que  $\left(k_{0},j\right)\in S_{i,j}$, $E_{i,j} = E_{i,k_{0}-1} + E_{k_{0},j}$\\
		Dans ce cas-là, le max est atteint:\\
		 $E_{i,k_{0}-1} + E_{k_{0},j} \geq  \displaystyle \max_{i<k<j}  E_{i,k-1} + E_{k,j}$ et $ \displaystyle \max_{i<k<j}  E_{i,k-1} + E_{k,j} \geq  E_{i,k_{0}-1} + E_{k_{0},j}$\\
		D'où \underline{ $E_{i,j} = E_{i,k_{0}-1} + E_{k_{0},j} = \displaystyle \max_{i<k<j}  E_{i,k-1} + E_{k,j}$}
		\end{description}
		En conclusion: $E_{i,j}$ est un majorant de l'ensemble $\{E_{i+1,j-1} + e(i,j), \, E_{i,j-1}, \,  \displaystyle \max_{i<k<j}  E_{i,k-1} \}$, et ce majorant est atteint: il y a toujours un cas où la condition d'égalité est vrai. 
		Donc
\begin{center}
 $E_{i,j} =\max\{E_{i+1,j-1} + e(i,j), \, E_{i,j-1}, \,  \displaystyle \max_{i<k<j}  E_{i,k-1} + E_{k,j}\}$
\par\end{center}
		...
	\subsubsection{}
	Si k = j, $ E_{k,j}=E_{j,j}=0$ par \ref{subsec:q1}, donc $E_{i,j-1}\in \{i<k\leq j \mid  E_{i,k-1} + E_{k,j}\} $, et 
\begin{center}
$E_{i,j-1}\leq \max\{i<k\leq j \mid  E_{i,k-1} + E_{k,j}\} $
\par\end{center}

\begin{center}
On en déduit que $E_{i,j} =\max\{E_{i+1,j-1} + e(i,j), \,   \displaystyle \max_{i<k<j}  E_{i,k-1} + E_{k,j}\}$
\par\end{center}

\subsection{}
	\begin{lstlisting}[frame=single]  % Start your code-block
	for i:=maxint to 0 do
	begin
	{ do nothing }
	end;
	Write('Case insensitive ');
	Write('Pascal keywords.');
	\end{lstlisting}


\end{document}
